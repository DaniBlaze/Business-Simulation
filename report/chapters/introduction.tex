The following report examines three different boarding procedures with the objective of finding a more efficient boarding policy. The background for the research is based on the views many people have today regarding the process of boarding a plane. Many travelers find the process inefficient and annoying, where they are spending a lot of time in queues. This involves wasting time at the gate and in the corridor of the airplane, waiting for people to store their luggage. In order to reduce costs associated with flight delays, the first two simulation methods will be analysed in terms of identifying potential bottlenecks and other elements of the boarding procedures which can be carried out in a more efficient way. This will lead to a third and final simulation method where the potential improvements from the first two will be implemented.   

\indent \newline 
The next chapters of the report will start by explaining how relevant data have been collected, certain assumptions which have been made and the build-up of the simulation methods. Different scenarios are then proposed, consisting of certain conditions the first two boarding methods will be simulated under. Further, performance metrics are described and results of the first two simulations presented, before going into the third simulation method. Lastly, recommendations will be given on how airlines can implement a more efficient boarding policy.   

